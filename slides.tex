\PassOptionsToPackage{usenames,dvipsnames}{xcolor}
\documentclass[xelatex,aspectratio=169]{beamer}

\usepackage{xifthen,multicol,textcomp,graphicx,minted}
\usepackage{fontspec}
\usepackage{tikz,amsmath,xmpp}
\usetikzlibrary{arrows,decorations.pathreplacing}
\newcommand*{\tikzmark}[1]{\tikz[overlay, remember picture] \coordinate ({#1});}

\usetheme{Atlassian}
\usecolortheme{whale}
\beamertemplatenavigationsymbolsempty

\title[]{Golang}
\subtitle{(get happy)}
\author[]{%
	Sam Whited\\*%
	{\tiny\texttt{\href{xmpp:sam@samwhited.com}{sam@samwhited.com}}}%
}
% \date{2015--12--18}
\titlegraphic{\includegraphics[width=20mm]{images/hipchat.png}\hspace*{.25\textwidth}\includegraphics[width=20mm]{images/atlassian.png}}

% Define a font family that supports IPA characters.
\usefonttheme{professionalfonts}
\usefonttheme{serif}
\newfontfamily\ipa{Charis SIL}
\newfontfamily\calluna{Calluna}
\newfontfamily\geotica{Geotica Three}
\setmainfont{Circular Pro}
\newfontfamily\dejavusansmono{DejaVu Sans Mono}
\setmonofont{DejaVu Sans Mono}
% \usepackage{newunicodechar}
% \newunicodechar{└}{\dejavusansmono{└}}
% \newunicodechar{├}{\dejavusansmono{├}}
% \newunicodechar{─}{\dejavusansmono{─}}
% \newunicodechar{│}{\dejavusansmono{│}}

\begin{document}

\begin{frame}
	\maketitle
\end{frame}

\begin{frame}
	\frametitle{The Zen of Go}
	\only<1>{\setbeamercolor{item}{fg=Violet}}
	\only<2>{\setbeamercolor{item}{fg=CheetoOrange}}
	\only<3>{\setbeamercolor{item}{fg=Mauve}}
	\only<4>{\setbeamercolor{item}{fg=Slate}}
	\only<5-6>{\setbeamercolor{item}{fg=LimeGreen}}
	\only<7>{\setbeamercolor{item}{fg=Brown}}
	\only<8>{\setbeamercolor{item}{fg=BrightPink}}
	\only<9>{\setbeamercolor{item}{fg=Cyan}}
	\begin{enumerate}
		\item<1> Familiarity admits brevity
		\item<2> Dependency analysis should be easy
		\item<3> Composition is better than inheritance
		\item<4> Static linking is better than dynamic linking
		\item<5-6> If it's worth complaining about, it's worth fixing
		\item<6> If it's not worth fixing, it's not worth mentioning
		\item<7> Semicolons are for parsers, not for people
		\item<8> Long term clarity is better than short term convenience
		\item<9> Do not communicate by sharing memory, share memory by communicating
	\end{enumerate}
	\vspace*{\fill}\hspace*{0.75\textwidth}
	\vspace*{-4em}
	\begin{tikzpicture}[remember picture,overlay]
		\only<1-8>{
		\node [xshift=-0.25\textwidth,yshift=2.5em] at (current page.south east)
		{\includegraphics[width=20mm]{images/gopher.png}};}
		\only<9>{
		\node [xshift=-0.25\textwidth,yshift=1.5em] at (current page.south east)
		{\includegraphics[width=20mm]{images/gopher.png}};}
	\end{tikzpicture}
\end{frame}

\begin{frame}
\def\firstcircle{(0,0) circle (2cm)}
\def\secondcircle{(60:2.5cm) circle (2cm)}
\def\thirdcircle{(0:2.5cm) circle (2cm)}
\begin{figure}[!h]
\centering
\begin{tikzpicture}
	\begin{scope}[shift={(3cm,-5cm)}, fill opacity=0.5]
		\fill[red] \firstcircle;
		\fill[green] \secondcircle;
		\fill[blue] \thirdcircle;
		\draw \firstcircle node[below] {Ease\hspace*{2em}};
		\draw \secondcircle node [above] {Compile time};
		\draw \thirdcircle node [below] {\hspace*{2em}Speed\\*(C)};
		\node [anchor=center] at (3.25em, 2em)
		{\includegraphics[width=7mm]{images/golang.png}};
	\end{scope}
\end{tikzpicture}
\end{figure}
\end{frame}

{
\renewcommand{\fcolorbox}[4][]{#4}
\begin{frame}[fragile]
	\frametitle{Declarations}
\begin{minted}{C}
// C
char *(*fp)( int, float *);
\end{minted}
\end{frame}
\begin{frame}[fragile]
	\frametitle{Declarations}
\begin{minted}{C}
     ╭────────────────────╮
     │ ╭───╮              │
     │ │╭─╮│              │
     │ │↑ ││              │
char *(*fp)( int, float *);
 ↑   ↑ ↑  ││              │
 │   │ ╰──╯│              │
 │   ╰─────╯              │
 ╰────────────────────────╯
\end{minted}
\end{frame}
\begin{frame}[fragile]
	\frametitle{Declarations}
\begin{minted}{go}
// Go
var fp *func(int64, *float64) (*byte)
\end{minted}
\end{frame}
\begin{frame}[fragile]
	\frametitle{Declarations}
\begin{minted}{go}
    ╭──╮ ╭──╮                  ╭─╮
    ┴  │ │  │                  │ ↓
var fp *func(int64, *float64) (*byte)
       │ ↑  │                  ↑
       ╰─╯  ╰──────────────────╯
\end{minted}
\end{frame}
\begin{frame}[fragile]
	\frametitle{Declarations}
\begin{minted}{C}
void (*signal(int, void (*fp)(int)))(int);
\end{minted}
\end{frame}
\begin{frame}[fragile]
	\frametitle{Declarations}
\begin{minted}{C}
      +-----------------------------+
      │                  +---+      │
      │  +---+           │+-+│      │
      │  ↑   │           │↑ ││      │
void (*signal(int, void (*fp)(int)))(int);
 ↑    ↑      │      ↑    ↑  ││      │
 │    +------+      │    +--+│      │
 │                  +--------+      │
 +----------------------------------+
\end{minted}
\end{frame}
\begin{frame}[fragile]
	\frametitle{Declarations}
\begin{minted}{go}
var signal func(int, *func(int)) *func(int)
\end{minted}
\end{frame}
}

{
\begin{frame}[fragile]
	\frametitle{Method Signatures}
\begin{minted}{go}
func (f *Foo) Location() (float64, float64, error)
\end{minted}
\end{frame}

\begin{frame}[fragile]
	\frametitle{Method Signatures}
\begin{minted}{go}
func (f *Foo) Location() (float64, float64, error)
func (f *Foo) Location() (lat,long float64, err error)
\end{minted}
\end{frame}
}

\begin{frame}[fragile]
	\frametitle{Example \lmss{№} 3}
\begin{minted}{go}
package main

import (
  "fmt"
)

func main() {
  fmt.Println("Hello Wörld")
}
\end{minted}
\end{frame}

\section[]{Concurrency}
\frame{\sectionpage}

\begin{frame}
	\frametitle{Concurrency}
	\let\thefootnote\relax

	\begin{flushleft}
		Like Erlang and friends, Go's concurrency model is based on CSP.
		\footnote{C. A. R. Hoare: Communicating Sequential Processes (CACM 1978)}
	\end{flushleft}

\end{frame}

\begin{frame}
	\frametitle{Concurrency}
	\centerline{\includegraphics{images/gophercomplex1.jpg}}
	\begin{flushleft}
		Lots of Gophers doing simple tasks such as incinerating their now-obsolete
		Python documentation (blatantly stolen from Rob Pike).\\
	\end{flushleft}
\end{frame}

\begin{frame}
	\frametitle{Parallelism is now easy}
	\centerline{\includegraphics{images/gophercomplex1.jpg}}
	\centerline{\includegraphics{images/gophercomplex1.jpg}}
\end{frame}

% \begin{frame}[fragile]
% 	\frametitle{Go Routines}
% \begin{columns}[fragile]
% \column[t]{0.5\textwidth}
% \centerline{Go}
% \begin{minted}{go}
% 	package main
% 
% 	func main() {
% 	  go func() {
% 	    select { } // Block forever
% 	  }()
% 	  println("A useless goroutine")
% 	}
% \end{minted}
% \column[t]{0.5\textwidth}
% \centerline{Unix}
% \begin{minted}{bash}
% 	./longrunningoperation &
% 	echo "Process backgrounded"
% \end{minted}
% \end{columns}
% \end{frame}

\begin{frame}
	\frametitle{Go Routines}
	\only<1>{\setbeamercolor{item}{fg=Violet}}
	\only<2>{\setbeamercolor{item}{fg=CheetoOrange}}
	\only<3>{\setbeamercolor{item}{fg=Mauve}}
	\only<4>{\setbeamercolor{item}{fg=Slate}}
	\only<5-6>{\setbeamercolor{item}{fg=LimeGreen}}
	\only<7>{\setbeamercolor{item}{fg=Brown}}
	\only<8>{\setbeamercolor{item}{fg=BrightPink}}
	\only<9>{\setbeamercolor{item}{fg=Cyan}}
	\begin{enumerate}
		\item<1> Goroutines are cheap (overhead limited to stack frame allocation)
		\item<2> Multiplexed onto threads as necessary
		\item<3> Goroutines do not block other goroutines
	\end{enumerate}
\end{frame}

\begin{frame}
	\Large
	\calluna
	\begin{figure}
	{\color{AshGray}%
		\tikzmark{begin}sam\tikzmark{atsep}@%
		\tikzmark{domainpart}\textcolor{Cyan}{SamWhited}\tikzmark{tld}.com\tikzmark{end}%
	}

		\tikz[overlay,remember picture] {
			\draw[decorate,decoration={brace,raise=5mm,amplitude=15pt}] (begin.north
			west) -- node [above=1.75em] {\small email \& jid} (end.north east) ;
			\draw[decorate,decoration={brace,raise=2mm,amplitude=5pt,mirror}]
			(begin.south west) -- node[below of=begin, below=-1.35em] {\tiny me} (atsep.south west) ;
			\draw[decorate,decoration={brace,raise=2mm,amplitude=5pt,mirror}]
			(domainpart.south west) -- node[below of=begin, below=-1.35em] {\tiny twitter
			\addfontfeature{Variant=1}{\&} bitbucket } (tld.south west) ;
			\draw[decorate,decoration={brace,raise=6mm,amplitude=10pt,mirror}]
			(domainpart.south west) -- node[below of=begin, below=-.125em] {\tiny blog} (end.south west) ;
		}
	\end{figure}
\end{frame}

\end{document}
